\chapter{История Музыкальных Сервисов}
\label{cha:ch_1}

\section{CD диски}

В начале истории компьютеров музыку можно было слушать, только используя CD диски~\footnote{(Compact Disk) --- оптический носитель информации в виде пластикового диска с отверстием в центре, процесс записи и считывания информации которого осуществляется при помощи лазера согласно~\cite{cd}}, что было очень не удобно, поскольку для возможности эти CD купить необходимо было ехать на концерты группы или в специальные магазины.

\section{Музыкальное цифровое пиратство}

В конце 1990-х -- начале 2000-х годов с появлением формата mp3~\footnote{(более точно, англ. MPEG-1/2/2.5 Layer 3 — это самый популярный формат файла для хранения аудиоинформации} стало стремительно развиватся цифровое пиратство~\footnote{безвозмездный обмен файлами в сети, которые принадлежат и продаются каким-либо правообладателем}.  Ежегодно появлялись крупные пиринговые сети~\footnote{компьютерная сеть, основанная на равноправии участников, каждый участник одновременно и сервер, и клиент} и пиратские сайты~\footnote{WEB-сайт занимающийся цифровым пиратством}.  Согласно книге~\cite{piracy}

\subsection{Napster}

Каждый пользователь Napster'а~\footnote{пиринговая сеть существовавшая с 1999 по 2001 год, обменивающаяся mp3 файлами} мог открыть доступ к своей медиатеке для всех и при этом бесплатно скачивать музыку у других пользовтелей сети.

Такой подход вместе с удобным (на то время) интерфейсом вызвал огромный ажиотаж.  Информация взята из книг~\cite{napster,piracy}

Такая огромная сеть, распростаняющая бесплатно музыку без рарешения авторов, не могла остаться не замечанной исполнителями, и в 2000-ом году на Napster подали в суд одновременно Metallica, Dr. Dre и RIAA~\footnote{Ассоциация звукозаписывающей индустрии Америки.  Одна из ведущих компаний борющихся против пиратства (из книги \cite{piracy})}. И Napster был закрыт.

\subsection{The Pirate Bay}

Самый известный торрент-трекер~\footnote{программа свыязывающая клиентов сети друг с другом, но напрямую не участвующая в обмене раздаваемых файлов} в мире The Pirate Bay появился в 2003 году. The Pirate Bay был проектом шведской НКО Piratbyrån ({}<<Пиратское бюро>>{}), которая поддерживала борьбу против копирайта и выступала за свободное распространение информации.

В отличие от Napster и других пиринговых сетей, The Pirate Bay начал использовать более эффективную систему торрентов, в которой файлы нужно было скачивать не целиком у одного пользователя, а по кусочкам у множества юзеров.

Запустившись как локальный скандинавский проект, The Pirate Bay быстро разросся по всему миру --- к концу 2004 года на трекере было уже более 60 тысяч файлов, а к концу 2005 года --- 2,5 миллиона активных пользователей согласно книге~\cite{piracy}

Однако, история судебных исков к проекту и блокировок сайта растянулись на несколько страниц текста книги~\cite{piracy} --- их серверы отключали от электропитания и вынуждали сайт переезжать на иностранные домены, а функционеров проекта даже сажали в тюрьму. В ряде государств The Pirate Bay заблокирован --- в том числе и в России. При этом в некоторых странах по причинам, которые сложно объяснить здравой логикой, The Pirate Bay работает и в 2022 году~\footnote{следующее зеркало The Pirate Bay https://www.pirateproxy.space работает при наличии VPN}.

\subsection{zaycev.net}

Napster и The Pirate Bay были сильно популярны среди зарубежных слушателей, однако пользовались малым спросом среди русских

В середине 2000-х в российском интернете было много сайтов, которые позволяли скачивать музыку нелегально, но появившийся в 2004 году пиратский сайт Zaycev.net быстро стал самым популярным среди них --- то ли из-за звучного названия, то ли из-за удобства интерфейса.

Аудиторией Zaycev.net был массовый слушатель. На сайте не было ни карточек артиста, ни возможности скачивать альбомы целиком, ни распределения контента по категориям, ни редкой музыки --- зато можно было быстро скачать радиохиты.

Однако после ужесточения антипиратского законодательства и стриминг-революции Zaycev.net не закрылись, а подстроились под новую реальность --- теперь это полноценный и абсолютно легальный стриминг-сервис с веб-версией и мобильными приложениями для iOS и Android, а также месячной подпиской за 169 рублей.

\section{Аудиостриминг}
Аудиостриминговая платформа --- это сайт/приложение, которое позволяет пользователям слушать музыку, не скачивая её, это достигается постоянным считыванием интернет соеденения, так что такие платформы требуют достаточно устойчивого интернет-соеденения

Музыкальный плейлист (далее просто плейлист) --- подборка аудиоконтента для воспроизведения на радио или с помощью медиаплеера


\subsection{Spotify}

Spotify был основан в 2006 году. И сейчас является самой популярной аудиостриминговой платформой в мире, что сказано в статье~\cite{cnn-best-music-platform}

В начале свого существования spotify покупает BitTorrent клиент~\footnote{приложение или сайт, позволяющий скачивать и раздавать файлы, используя torrent сеть} <<\(\mu\)Torrent>> и использует собственный алгоритм, позволяющий слушать mp3 файлы из сети torrent~\footnote{spotify использовал сеть The Pirate Bay, которая как и spotify родом из Швеции}, который тоже находи, не скачивая файлов. Хоть эта функция тратила много оперативной памяти и батареи устройств, глупо отрицать что это было для пользователей тех лет как минимум в новинку.  Сейчас spotify отказался от использования torrent сетей.  Информация взята из статьи~\cite{spotify-success-story}

Чтобы не повторить судьбу Napster-а или того же The Pirate Bay, spotify заключает с исполнителями контракты.  Чтобы найти деньги на них spotify-ю приходится ввести в своё приложение premium подписку за которую пользователи должны были ежемесячно платить

Ещё одна причина популярности spotify, заключается в том, что пользователь spotify мог находить новую для себя музыку внутри spotify, используя плейлисты собранные специальными людьми, разбирающихся в музыке, или с помощью нейронных сетей

Spotify появился на российском рынке в 2020-ом году, а в 2022-ом ушёл из-за сложившейся политической ситуации

\subsection{Музыка ВКонтакте}

Социальная сеть ВКонтакте был основана в 2006 году.  Уже тогда она обладала функциями музыкальной платформой.  В VK~\footnote{(сокращение от В Контакте) --- популярная в россии социальная сеть} любой желающий мог выложить свои треки, что дало сильный толчок для цифрового пиратства в VK.  После ужесточения антипиратского законодательства <<VK Музыка>> легализовала музыкальные композиции в 2016 году.

В отличие от других музыкальных сервисов VK в первою очередь является социальной сетью, так что явным достоинством является интегрирование с социальной сетью.  Однако алгоритм подбора музыки является одним из самых плохих

\subsection{Звук}

Zvooq~\footnote{в прошлом название <<Звука>>} основался в 2010 году.  Пиком Zvooq'а был 2014 год, когда он занял первое место среди всех других приложений AppStore~\footnote{магазин приложений для устройств Apple} согласно статье~\cite{zvooq-first-in-appstore}.  С 2020 года является собственностью <<Сбербанка>>

\subsection{Yandex.Music}

После ухода spotify c российского рынка Yandex.Music стал самой популярной заменой spotify в России согласно статье~\cite{ixbt-yandex-music}, хоть вместе с spotify с российского рынка ушёл и крупный правообладатель музыки {}<<Sony Music>>{}, который сотрудничал с Yandex.Music, из за чего эта платформа, как и другие российские музыкальные сервисы, потеряла часть своей музыки

Если сравнивать spotify с Yandex.Music, то spotify будет выигрывать в разнобразии зарубежной музыки, а Yandex.Music в разнобразии отечественной

%%% Local Variables:
%%% mode: latex
%%% TeX-master: "0-main"
%%% End:
