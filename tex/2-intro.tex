\Introduction

В процессе написания исследовательской работы по информатике были проанализированы истории нескольких музыкальных площадок разных времён, разработана программа, помагающая любителям музыки искать музыкальные произведения, пользуясь более одной музыкальной базой (например использовать Spotify, для качественного поиска информации о треках: год релиза, длительность, а rocknation~\footnote{сайт с бесплатным доступом к произведением в жанре rock} только для скачивания)

\subsection{Название Проекта}
Приложение для Управления Библиотекой Музыкой

\subsection{Выполнявший}
Храмцов Семён Михайлович

\subsection{Руководитель проекта}
Храмцова Евгения Алексеевна

\subsection{Противоречие}

В современном мире существует много музыкальных платформ, в чём-то лучше других, в чём-то хуже других, и типичному пользователю неудобно искать треки, используя несколько платформ одновременно.

\subsection{Цель}
Создать к концу учебного года программу для быстрого и удобного
объеденения данных со всех музыкальных платформ в одну медиатеку

\subsection{Задачи проекта}

\begin{itemize}
\item
  Проанализировать историю развития методов прослушивания музыки
\item
  Проанализировать достоинства и недостатки современных музыкальных платформ
\item
  Создать программу, позволяющую искать музыкальные произведения, используя несколько музыкальных сервисов
\end{itemize}

\subsection{Сильные стороны}

\begin{itemize}
\item Гибкость.  Возможность добавлять поддержку любых музыкальных сервисов, мало изменяя код
\item Открытость.  Код расположен на GitHub
\end{itemize}

\subsection{Слабые стороны}

\begin{itemize}
\item Неинтуитивно-понятный интерфейс в виде telegram бота
\item
  Доступ к программе есть только, когда включён мой компьютер, стоит арендовать сервер
\item
  Изначально дана поддержка малого количества музыкальных платформ  (на данный момент: spotify, rocknation)
\end{itemize}

\subsection{Возможные эффекты}

\subsubsection{Социальный}
Помощь учащимся в поисках музыкальных треков

\subsubsection{Технологический}
Соеденение нескольких музыкальных платформ для поиска треков

\subsubsection{Финансовый}

\begin{enumerate}
\item
  Программа бесплатна и в открытом доступе \(\implies\) экономия по
  сравнении с другими программами данного типа

\item
  Музыкальными платформами также могут быть и бесплатные источники, так, что уменьшается необходимость в премиум аккаунтах, таких приложений, как Yandex.Music и spotify
\end{enumerate}
